% --------------------------------------------------------
% DEFINIÇÕES DO DOCUMENTO
% --------------------------------------------------------

\documentclass[
	% -- opções da classe memoir --
	12pt,				% tamanho da fonte
	openright,			% capítulos começam em pág ímpar (insere página vazia caso preciso)
	oneside,			% para impressão em verso e anverso. Oposto a twoside
	a4paper,			% tamanho do papel.
	% -- opções da classe abntex2 --
	%chapter=TITLE,		% títulos de capítulos convertidos em letras maiúsculas
	%section=TITLE,		% títulos de seções convertidos em letras maiúsculas
	%subsection=TITLE,	% títulos de subseções convertidos em letras maiúsculas
	%subsubsection=TITLE,% títulos de subsubseções convertidos em letras maiúsculas
	% -- opções do pacote babel --
	english,			% idioma adicional para hifenização
	french,				% idioma adicional para hifenização
	spanish,			% idioma adicional para hifenização
	brazil,				% o último idioma é o principal do documento
	]{lib/abntex2}


% --------------------------------------------------------
% PACOTES
% --------------------------------------------------------

\usepackage{cmap}				% Mapear caracteres especiais no PDF
\usepackage{lmodern}			% Usa a fonte Latin Modern
\usepackage[T1]{fontenc}		% Selecao de codigos de fonte.
\usepackage[utf8]{inputenc}		% Codificacao do documento (conversão automática dos acentos)
\usepackage{lastpage}			% Usado pela Ficha catalográfica
\usepackage{indentfirst}		% Indenta o primeiro parágrafo de cada seção.
\usepackage{color}				% Controle das cores
\usepackage{graphicx}			% Inclusão de gráficos
\usepackage{lipsum}				% para geração de dummy text
\usepackage[brazilian,hyperpageref]{}

\usepackage{silence}
%Disable all warnings issued by latex starting with "You have..."
\WarningFilter{latex}{You have requested package}
\usepackage[alf]{lib/abntex2cite}	% Citações padrão ABNT
\usepackage[br]{lib/nicealgo}       % Pacote para criação de algoritmos
\usepackage{lib/customizacoes}      % Pacote de customizações do abntex2


% --------------------------------------------------------
% CONFIGURAÇÕES DE PACOTES
% --------------------------------------------------------

% Configurações do pacote backref
\renewcommand{\familydefault}{\sfdefault}
% Usado sem a opção hyperpageref de backref
% \renewcommand{\backrefpagesname}{Citado na(s) página(s):~}
% Texto padrão antes do número das páginas
% \renewcommand{\backref}{}
% Define os textos da citação
% \renewcommand*{\backrefalt}[4]{
% 	\ifcase #1 %
% 		Nenhuma citação no texto.%
% 	\or
% 		Citado na página #2.%
% 	\else
% 		Citado #1 vezes nas páginas #2.%
% 	\fi}%


% --------------------------------------------------------
% INFORMAÇÕES DE DADOS PARA CAPA E FOLHA DE ROSTO
% --------------------------------------------------------

\titulo{Título do Texto}
\autor{Nome do Autor}
\newcommand{\RA}{0123456789}
\local{Bauru}
\data{Maio/2015}
\orientador{Prof. Dr. Seu Orientador}
\instituicao{%
  Universidade Estadual Paulista ``Júlio de Mesquita Filho''
  \par
  Faculdade de Ciências
  \par
  Ciência da Computação}
\tipotrabalho{Proposta para Trabalho de Conclusão de Curso}
\preambulo{Proposta para Trabalho de Conclusão de Curso do Curso de Bacharelado em Ciência da Computação da Universidade Estadual Paulista ``Júlio de Mesquita Filho'', Faculdade de Ciências, Campus Bauru.}


% --------------------------------------------------------
% CONFIGURAÇÕES PARA O PDF FINAL
% --------------------------------------------------------

% alterando o aspecto da cor azul
\definecolor{blue}{RGB}{41,5,195}

% informações do PDF
\makeatletter
\hypersetup{
  %pagebackref=true,
  pdftitle={\@title},
  pdfauthor={\@author},
  pdfsubject={\imprimirpreambulo},
  pdfcreator={LaTeX with abnTeX2},
  pdfkeywords={abnt}{latex}{abntex}{abntex2}{trabalho acadêmico},
  colorlinks=true,    % false: boxed links; true: colored links
  linkcolor=black,    % color of internal links
  citecolor=black,    % color of links to bibliography
  filecolor=magenta,  % color of file links
  urlcolor=black,
  bookmarksdepth=4
}
\makeatother


% --------------------------------------------------------
% ESPAÇAMENTOS ENTRE LINHAS E PARÁGRAFOS
% --------------------------------------------------------

% O tamanho do parágrafo é dado por:
\setlength{\parindent}{1.3cm}

% Controle do espaçamento entre um parágrafo e outro:
\setlength{\parskip}{0.2cm}


% --------------------------------------------------------
% COMPILANDO O ÍNDICE
% --------------------------------------------------------

\makeindex


% --------------------------------------------------------
% INÍCIO DO DOCUMENTO
% --------------------------------------------------------

\begin{document}

% Seleciona o idioma do documento (conforme pacotes do babel)
\selectlanguage{brazil}

% Retira espaço extra obsoleto entre as frases.
\frenchspacing


% --------------------------------------------------------
% ELEMENTOS PRÉ-TEXTUAIS
% --------------------------------------------------------

% Capa
\imprimircapaproposta

% Folha de rosto
% (o * indica que haverá a ficha bibliográfica)
\imprimirfolhaderosto


% resumo em português
\setlength{\absparsep}{18pt} % ajusta o espaçamento dos parágrafos do resumo
\begin{resumo}
Espaço destinado à escrita do resumo.
\textbf{Palavras-chave:} Palavras-chave de seu resumo.
\end{resumo}

% --------------------------------------------------------
% SUMÁRIO
% --------------------------------------------------------

% inserir o sumario
\pdfbookmark[0]{\contentsname}{toc}
\tableofcontents*
\cleardoublepage


% --------------------------------------------------------
% ELEMENTOS TEXTUAIS
% --------------------------------------------------------

\pagestyle{simple}

% Arquivos .tex do texto, podendo ser escritos em um único arquivo ou divididos da forma desejada
\chapter{Introdução}
\label{c.introducao}

\section{Problema}
\label{s.problema}

\section{Proposta}
\label{s.proposta}

\section{Objetivos}
\label{s.objetivo}

\subsection{Objetivo Geral}
\label{ss.objetivo_geral}

\subsection{Objetivos Específicos}
\label{ss.objetivo_especifico}

\section{Estrutura da Dissertação}
\label{s.estrutura}
\chapter{Conclusão}
\label{c.conclusao}

\section{Direcionamentos para Trabalhos Futuros}
\label{s.future_work}

\section{Publicações Realizadas}
\label{s.publication}

% --------------------------------------------------------
% REFERÊNCIAS BIBLIOGRÁFICAS
% --------------------------------------------------------

\bibliography{text/chapters/referencias}


% --------------------------------------------------------
% ÍNDICE REMISSIVO
% --------------------------------------------------------

\printindex


% --------------------------------------------------------
% FINAL DO DOCUMENTO
% --------------------------------------------------------

\end{document}
